
\documentclass[11pt,a4paper]{article}
\usepackage{float}
% Packages needed
\usepackage[utf8]{inputenc}
\usepackage[T1]{fontenc}
\usepackage{amsmath, amssymb, amsthm}
\usepackage{mathrsfs}
\usepackage{tikz}       % For drawing the diagram
\usetikzlibrary{arrows.meta, decorations.pathreplacing}
\usetikzlibrary{patterns}
\usepackage[french]{babel} % Because the text is French (optional, or remove if not needed)

\begin{document}
	
	\section*{V.D}
	
	\subsection*{I. Variété topologique}
	
	Soit $M$ un espace topologique séparé.
	
	\subsubsection*{1. Carte}
	
	\textbf{Définition.} Une carte locale (sur $M$ en $p$) est un couple $(U, \varphi)$, où $U$ est un ouvert contenant $p$.
	
	\[
	\varphi: U \to \varphi(U) \subset \mathbb{R}^n
	\]
	où $\varphi$ est un homéomorphisme.
	
	\begin{center}
		\begin{tikzpicture}[scale=1.1]
			% Draw M as an ellipse
			\draw[thick] (0,0) ellipse (2 and 1.2);
			\node at (2.4,1.1) {$M$};
			
			% Draw U inside
			\draw[thick] (0,-0.1) ellipse (0.7 and 0.4);
			% Fill U for illustration
			\fill[gray!20,opacity=0.6] (0,-0.1) ellipse (0.7 and 0.4);
			\node at (0,-0.6) {$U$};
			
			% Point p inside U
			\fill (0,-0.1) circle (2pt);
			\node[anchor=west] at (0.1,-0.1) {$p$};
			
			% Arrow downward from p
			\draw[->,thick] (0,-0.1) -- (0,-2);
			\node at (-0.2,-1) {$\varphi$};
			
			% Draw phi(U) as ellipse in R^n
			\draw[thick] (0,-2.6) ellipse (1 and 0.5);
			\fill[gray!20,opacity=0.6] (0,-2.6) ellipse (1 and 0.5);
			\node at (0,-3.25) {$\varphi(U) \subset \mathbb{R}^n$};
			
			% Draw phi(p)
			\fill (0,-2.6) circle (2pt);
			\node[anchor=west] at (0.1,-2.6) {$\varphi(p)$};
		\end{tikzpicture}
	\end{center}
	
	\underline{\textbf{Exemple.}} Soit $E$ un e.v. de dimension $n$ et $\{e_1, \ldots, e_n\}$ une base de $E$. \\
	Pour tout $v \in E$, on a :
	\[
	v = \sum_{i=1}^n v_i e_i.
	\]
	On considère
	\[
	\varphi : E \longrightarrow \mathbb{R}^n, \qquad v \longmapsto (v_1, \ldots, v_n)
	\]
	$(E, \varphi)$ est une carte.
	
%\begin{figure}[]
%		\begin{center}
%		\begin{tikzpicture}[scale=1.4, every node/.style={font=\small}]
%			
%			% Draw M as an irregular outer ellipse
%			\draw[thick] (0,2.4) .. controls (2,3.7) and (5,3.5) .. (4.7,2)
%			.. controls (4.5,0.6) and (2.5,0.1) .. (1,0.3)
%			.. controls (-0.5,0.6) .. (0,2.4);
%			
%			\node at (4.8,2.7) {$M$};
%			
%			% Draw U1 and U2
%			\draw[thick] (1,1.6) ellipse (1.1 and 0.7);
%			\node at (0.1,2.2) {$U_1$};
%			
%			\draw[thick] (3.1,1.7) ellipse (1.1 and 0.7);
%			\node at (4,2.3) {$U_2$};
%			
%			% Shaded intersection
%			\begin{scope}
%				\clip (1,1.6) ellipse (1.1 and 0.7);
%				\fill[pattern=north east lines, opacity=0.3]
%				(3.1,1.7) ellipse (1.1 and 0.7);
%			\end{scope}
%			
%			% phi_1 arrow and its image
%			\draw[thick, ->] (0.5,1) to[bend right=30] (-0.8,0.2);
%			\node at (-1.1,0.5) {$\varphi_1$};
%			\draw[thick] (-1.3,-0.3) ellipse (0.65 and 0.35);
%			\node at (-1.3,-0.7) {$\varphi_1(U_1)$};
%			
%			% Shaded part in \varphi_1(U_1)
%			\fill[pattern=north east lines, opacity=0.3] 
%			(-1.3,-0.3) ellipse (0.3 and 0.15);
%			
%			% phi_2 arrow and its image
%			\draw[thick, ->] (3.4,1) to[bend left=30] (5,0.2);
%			\node at (5.25,0.5) {$\varphi_2$};
%			\draw[thick] (5.3,-0.3) ellipse (0.65 and 0.35);
%			\node at (5.3,-0.7) {$\varphi_2(U_2)$};
%			
%			% Shaded part in \varphi_2(U_2)
%			\fill[pattern=north east lines, opacity=0.3] 
%			(5.3,-0.3) ellipse (0.3 and 0.15);
%			
%			% R^n axes
%			\draw[thick, ->] (2,-0.2) -- (2,1);
%			\draw[thick, ->] (2,-0.2) -- (0.3,-1.1);
%			\draw[thick, ->] (2,-0.2) -- (3.7,-1.1);
%			\node at (2,-0.5) {$\mathbb{R}^n$};
%			
%			% transition map arrow
%			\draw[thick, ->] (-0.7,0) .. controls (0.2,-0.7) and (3.8,-0.7) .. (4.8,0);
%			\node at (2,-1.05) {$\varphi_2 \circ \varphi_1^{-1}$};
%		\end{tikzpicture}
%		
%		\end{center}
%\end{figure}
	
	\underline{\textbf{3- Atlas}}
	
	\medskip
	
	\underline{\textbf{Déf.}} Un atlas sur $M$ est la donnée sur $M$ d'une famille
	\[
	\mathcal{A} = \left\{ (U_i, \varphi_i) \right\}_{i \in I}
	\]
	de cartes telle que
	\[
	M = \bigcup_{i \in I} U_i.
	\]
	\textit{Si toutes les cartes sont de même dimension $n$, on dit que \ldots}
	% (You can complete the sentence as needed)
	
\newpage	
	\underline{\textbf{Ex (Sphère $\mathbb{S}^2$)}}
	
	\[
	\mathbb{S}^2 = \left\lbrace (x_1, x_2, x_3) \in \mathbb{R}^3 ~\Big|~ \sum_{i=1}^3 x_i^2 = 1 \right\rbrace
	\]
	avec la topologie induite.\\
	Pour $i=1,2,3$, on pose :
	\[
	U_i^+ = \left\lbrace (x_1, x_2, x_3) \in \mathbb{R}^3 ~|~ x_i > 0 \right\rbrace
	\]
	\[
	U_i^- = \left\lbrace (x_1, x_2, x_3) \in \mathbb{R}^3 ~|~ x_i < 0 \right\rbrace
	\]
	
	\[
	\begin{array}{ccc}
		\varphi_i^+ : U_i^+ & \longrightarrow & \mathbb{R}^2 \\
		\quad\ x & \longmapsto & \check{x}^{i} \\
	\end{array}
	\qquad
	\begin{array}{ccc}
		\varphi_i^- : U_i^- & \longrightarrow & \mathbb{R}^2 \\
		\quad\ x & \longmapsto & \check{x}^{i} \\
	\end{array}
	\]
	\vspace{-1em}
	où, par exemple, si $x=(x_1, x_2, x_3)$, alors
	\[
	\check{x}^{\,2} = (x_1, x_3).
	\]
	\[
	{\color{red} U_1^+ = \left\{ (x_1, x_2, x_3) \;\middle|\; x_1>0 \right\} }
	\]
	
	\textbf{Montrer que :}
	\begin{enumerate}
		\item $U_i^+,\, U_i^-$ sont des ouverts de $\mathbb{S}^2$;
		\item $\varphi_i^+(U_i^+) = \varphi_i^-(U_i^-) = D_2$ (disque unité);
		\item $\varphi_i^+,\, \varphi_i^-$ sont des homéomorphismes sur $D_2$;
		\item $\left\{ (U_i^+, \varphi_i^+),\, (U_i^-, \varphi_i^-) \right\}$ forme un atlas sur $\mathbb{S}^2$;
		\item Écrire les fonctions de transition.
	\end{enumerate}
	
\newpage 
	\underline{\textbf{Proposition}}: Soient $M$ et $N$ deux variétés topologiques et $ f \colon M \rightarrow N $, une application et $x_0 \in M$. 
	
	Les assertions suivantes sont equivalentes:
	
	\begin{enumerate}
		\item $f$ est continue en $x_0$.
		\item Il existe $(U, \varphi)$ en $x_0$, $(V, \psi)$ en $f(x_0)$ avec $f(U) \subset V$ et
		\[
		\psi \circ f \circ \varphi^{-1} : \varphi(U) \to \psi(V)
		\]
		est continue en $\varphi(x_0)$.
		\item Pour tous $(U, \varphi)$ en $x_0$, $(V, \psi)$ en $f(x_0)$ avec $f(U) \subset V$, l'application
		\[
		\psi \circ f \circ \varphi^{-1} : \varphi(U) \to \psi(V)
		\]
		est continue en $\varphi(x_0)$.
	\end{enumerate}
	
	\begin{figure}[H]
		\begin{center}
			\begin{tikzpicture}[>=latex,scale=1.2]
				% Top nodes: M and N with open sets U,V
				\node (M) at (-1.7,3.7) {$M$};
				\draw (0,3.7) ellipse (1.4 and 1);
				\node (U) at (-0.5,4.2) {$U$};
				\draw (-0.5,3.7) ellipse (0.5 and 0.35);
				\node (x0) at (-0.5,3.7) [circle,fill=black,inner sep=1.2pt]{};
				\node at (-0.68,3.52) {$x_0$};
				%
				\node (N) at (4.5,3.7) {$N$};
				\draw (6,3.7) ellipse (1.4 and 1);
				\node (V) at (5.7,3.7) {$V$};
				\draw (6.5,3.7) ellipse (0.5 and 0.35);
				\node (fx0) at (6.5,3.7) [circle,fill=black,inner sep=1.2pt]{};
				\node at (6.65,3.52) {$f(x_0)$};
				
				% f: M -> N
				\draw[->,thick] (1.0,3.85) .. controls (3,5.0) and (4,5.0) .. (5.0,3.85) node[midway,above=10pt] {$f$};
				
				% Left chart: phi: U -> R^m
				\draw[->,thick] (-0.65,3.2) .. controls (-2.,2.4) and (-0.9,1.2) .. (0,0.5);
				\node at (-1.2,1.4) {$\varphi$};
				
				% Right chart: psi: V -> R^n
				\draw[->,thick] (6.65,3.2) .. controls (8.,2.5) and (7.,1.) .. (6,0.5);
				\node at (7.1,1.4) {$\psi$};
				
				% R^m and R^n axes
				\draw[->] (0,0.5) -- ++(0,2);
				\draw[->] (0,0.5) -- ++(-0.9,-0.6);
				\draw[->] (0,0.5) -- ++(0.9,-0.6);
				\node at (0,-0.1) {$\mathbb{R}^m$};
				
				\draw[->] (6,0.5) -- ++(0,2);
				\draw[->] (6,0.5) -- ++(-0.9,-0.6);
				\draw[->] (6,0.5) -- ++(0.9,-0.6);
				\node at (6,-0.1) {$\mathbb{R}^n$};
				
				% Images of U, V under charts and points
				\draw (0,0.5) ellipse (0.5 and 0.21);
				\node at (0.7,0.8) {$\varphi(U)$};
				\draw[pattern=north east lines,opacity=0.3] (0,0.5) ellipse (0.23 and 0.1);
				\draw (0,0.5) [fill=black] circle (.05);
				\node at (-0.17,0.32) {$\varphi(x_0)$};
				
				\draw (6,0.5) ellipse (0.5 and 0.21);
				\node at (5.25,0.8) {$\psi(V)$};
				\draw[pattern=north east lines,opacity=0.3] (6,0.5) ellipse (0.23 and 0.1);
				\draw (6,0.5) [fill=black] circle (.05);
				\node at (6.13,0.32) {$\psi(f(x_0))$};
				
				% Lower map: psi o f o phi^{-1}
				\draw[->,thick] (0.5,0.5) -- (5.5,0.5);
				\node at (3,0.68) {$\psi \circ f \circ \varphi^{-1}$};
			\end{tikzpicture}
		\end{center}
	\end{figure}
	
	\section{04.07.2025}
	\underline{\textbf{Exercice sur $S^3 \subset \mathbb{R}^4$}}
	
	\[
	X = -x^2 \frac{\partial}{\partial x^1} + x^1 \frac{\partial}{\partial x^2} + x^4 \frac{\partial}{\partial x^3} - x^3 \frac{\partial}{\partial x^4}
	\]
	\[
	Y = -x^3 \frac{\partial}{\partial x^1} - x^4 \frac{\partial}{\partial x^2} + x^1 \frac{\partial}{\partial x^3} + x^2 \frac{\partial}{\partial x^4}
	\]
	\[
	Z = -x^4 \frac{\partial}{\partial x^1} + x^3 \frac{\partial}{\partial x^2} - x^2 \frac{\partial}{\partial x^3} + x^1 \frac{\partial}{\partial x^4}
	\]
	
	\begin{enumerate}
		\item S'assurer que ce sont des champs de vecteurs sur $S^3$.
		\item Montrer qu'ils sont linéairement indépendants.
	\end{enumerate}
	

2. \textbf{Crochet de Lie de Champ de vecteur}
Soient 
\[
X = \sum_{i=1}^{n} X^i \frac{\partial}{\partial x^i} = X^i \partial_i
\]
et
\[
Y = \sum_{j=1}^{n} Y^j \frac{\partial}{\partial x^j} = Y^j \partial_j
\]
où\quad $\partial_k = \frac{\partial}{\partial x^k}$
	
	
\begin{align*}
	TM &= \bigcup_{p \in M} T_pM \\ 
	&= \left. \bigcup_{p \in M} T_pM \right. \\ 
		&\equiv \text{Fibré tangent de } M\\
\end{align*}

	
	Comme $T_pM$ est un e.v., on note $T_p^*M$ son dual. On pose
	\[
	T^*M = \bigcup_{p \in M} T_p^*M
	\]
	\[
	\text{(Fibré cotangent.)}
	\]

		On définit une projection naturelle~: 
	\begin{align*}
		\pi &: TM \to M \\ 
		(p, v_p) &\mapsto p, \text{ si } v_p \in T_p M
	\end{align*}
		
		\underline{Def.} Un champ de vecteurs sur $M$ est une application 
		\[
		X : M \to TM
		\]
		vérifiant :
		\[
		\pi \circ X = \mathrm{id}_M
		\]
		
		\textcolor{red}{(En fait $TM$ est une variété de dimension $2 \dim M$)}
		
		
		Si $X$ est un champ de vecteurs sur $M$, $X(p) \in T_pM$, \\
		$\forall p \in M$, donc en c.l.
		\[
		X(p) = \sum_{i=1}^{n} X^i(p) \left. \frac{\partial}{\partial x^i} \right|_p
		\]
		On peut donc écrire :
		\[
		X = \sum_{i=1}^{n} X^i \frac{\partial}{\partial x^i}
		\]
		
		
		où $X^i$ sont des fonctions lisses sur un ouvert de $M$.
		
		\underline{\textbf{Notation}}
		
		\begin{itemize}
			\item $\mathscr{X}(M)$ = champs lisses sur $M$.
			\item Si $X \in \mathscr{X}(M)$, $p \in M$, $X_p := X(p)$.
			\item Si $f : M \longrightarrow N$ est différentiable en $p \in M$ et $X \in \mathscr{X}(M)$,
		\end{itemize}
		
		\[
		T_p f \cdot X|_p = X|_p(f)
		\]
		désignera l'action de la différentielle en $p$ de $f$ sur le vecteur tangent $X|_p$.
		
		\[
		\text{\underline{Ex}} \qquad \text{Sur } \mathbb{R}^3
		\]
		\[
		X = xz \frac{\partial}{\partial x} + x^2 y \frac{\partial}{\partial y} + \frac{\partial}{\partial z}
		\]
		
		\underline{\textbf{Exercice}} sur $S^3 \subset \mathbb{R}^4$, \\
		
		\[
		X = -x^2 \frac{\partial}{\partial x^1} + x^1 \frac{\partial}{\partial x^2} + x^4 \frac{\partial}{\partial x^3} - x^3 \frac{\partial}{\partial x^4}
		\]
		\[
		Y = -x^3 \frac{\partial}{\partial x^1} - x^4 \frac{\partial}{\partial x^2} + x^1 \frac{\partial}{\partial x^3} + x^2 \frac{\partial}{\partial x^4}
		\]
		\[
		Z = -x^4 \frac{\partial}{\partial x^1} + x^3 \frac{\partial}{\partial x^2} - x^2 \frac{\partial}{\partial x^3} + x^1 \frac{\partial}{\partial x^4}
		\]
		
		\vspace{0.3cm}
		\begin{enumerate}
			
			\item S'assurer que ce sont des champs de vecteurs sur $S^3$.
			\item Montrer qu'ils sont linéairement indépendants.
			
		\end{enumerate}
		
\begin{align*}
	XY &= \left( \sum_{i=1}^n X^i \frac{\partial}{\partial x^i} \right) \left( \sum_{j=1}^n Y^j \frac{\partial}{\partial x^j} \right) \\ 
	&= \sum_{i=1}^n \sum_{j=1}^n \left( X^i \frac{\partial}{\partial x^i} \right) \left( Y^j \frac{\partial}{\partial x^j} \right) \\
	&= \sum_{i=1}^n \sum_{j=1}^n X^i \left[ \frac{\partial Y^j}{\partial x^i} \frac{\partial}{\partial x^j} + Y^j \frac{\partial^2}{\partial x^i \partial x^j} \right] \\
	&= \left( X^i \frac{\partial Y^j}{\partial x^i} \right) \frac{\partial}{\partial x^j} + X^i Y^j \frac{\partial^2}{\partial x^i \partial x^j}
\end{align*}
	
	\begin{align*}
		\cdot \quad YX &= (Y^j \partial_j X^i) \partial_i + X^i Y^j \partial_{ji}^2 \\
		\cdot \quad XY - YX &= \left[ X^i \partial_i Y^j - Y^i \partial_i X^j \right] \partial_j
	\end{align*}
	On pose \([X, Y] = XY - YX =\) \emph{crochet de Lie de \(X\) et \(Y\)}.
	
	\section{07.07.2025}
	Pour tout $g \in G$, on définit \\
	$L_g : G \to G$, $R_g: G \to G$\\
	\[
	\begin{array}{ll}
		L_g : & h \mapsto gh \\
		R_g : & h \mapsto hg
	\end{array}
	\]
	Elles sont $C^\infty$ et vérifient :
	\[
	\begin{aligned}
		& L_g \circ R_h = R_h \circ L_g, \quad \forall g, h \\
		& L_g \circ L_h = L_{gh} \\
		& R_g \circ R_h = R_{hg} \\
		& L_e = R_e = Id_G \\
	\end{aligned}
	\]
	\medskip
	
	\noindent
	\textasteriskcentered~Les translations sont des difféo. \\
	$L_{g^{-1}} = L_g^{-1}$ \\
	et $R_{g^{-1}} = R_g^{-1}$ \\
	
	\medskip
	
	\noindent
	\textbf{Exemple} \\
	$GL(n, \mathbb{R})$ \\
	\[
	H_3 = \left\lbrace
	\begin{pmatrix}
		1 & x & z \\
		0 & 1 & y \\
		0 & 0 & 1
	\end{pmatrix} ,\, x, y, z \in \mathbb{R}
	\right\rbrace
	\]
	\medskip
	
	\noindent
	\textasteriskcentered~On rappelle qu'une transformation affine de $\mathbb{R}$ est une application
	\[
	f : \mathbb{R} \longrightarrow \mathbb{R}, \quad x \mapsto ax + b
	\]
	où $(a, b) \in \mathbb{R}^* \times \mathbb{R}$.
	\\
	Si $g : x \mapsto a'x + b'$ est une autre transformation affine, on a :
	\[
	(f \circ g)(x) = f(a'x + b') = a(a'x + b') + b = aa'x + ab' + b
	\]
	\medskip
	
	$f \circ g$ est un T.A.
	
	\[
	\text{Aff}(\mathbb{R}) = \left\lbrace f : x \mapsto ax + b,\,\, a \in \mathbb{R}^*, b \in \mathbb{R} \right\rbrace
	\]
	est un groupe.
	
	L'inverse $f : x \mapsto a x + b$ est
	\[
	f^{-1} : x \mapsto \frac{1}{a}x - \frac{b}{a}
	\]
	En fait $\operatorname{Aff}(\mathbb{R}) = \mathbb{R}^* \ltimes \mathbb{R}$. Avec la structure différentielle induite par celle de $\mathbb{R}^* \times \mathbb{R}$, $\operatorname{Aff}(\mathbb{R})$ est un groupe de Lie.
	
	\[
	\operatorname{Aff}(\mathbb{R}) = \mathbb{R}^* \times \mathbb{R}
	\]
	avec la loi
	\[
	(a, b) (a', b') = (a a', a b' + b)
	\]
	
	\textbf{Déf.} Soit $G$ un groupe de Lie.\\
	Un sous-groupe de Lie de $G$ est une partie $H$ de $G$ vérifiant :
	\begin{itemize}
		\item $H$ est un sous-groupe de $G$,
		\item $H$ est une sous-variété immergée de $G$ : $H \hookrightarrow G$ est une immersion.
	\end{itemize}
	
	\textbf{\underline{Thm (É. Cartan)}}
	
	\vspace{1em}
	
	Tout sous-groupe fermé d'un groupe de Lie est un sous-groupe de Lie.
	
	\vspace{2em}
	
	En conséquence, tout sous-groupe fermé de $GL_n(\mathbb{R})$ est un groupe de Lie, dit groupe de Lie linéaire.
	
	\underline{\textbf{Exemple}}
	
	\begin{itemize}
		\item $SL_n(\mathbb{R})$, $SL_n(\mathbb{C})$
		\item $O(n)$, $O(n,\mathbb{C})$
		\item $SO(n)$, $U(n)$
	\end{itemize}
	
	sont des groupes de Lie linéaires.
	
	\vspace{1em}
	\[
	\underline{\text{III. \; Algèbre de Lie d'un groupe de Lie.}}
	\]
	
	
	\textbf{1. Champ de vecteurs invariant à gauche.}
	
	\vspace{1em}
	
	$G$, $e$, $\mathfrak{X}(G)$ (algèbre de Lie)
	
	\vspace{1em}
	
	\underline{Déf.} $X \in \mathfrak{X}(G)$ est dit invariant à gauche si :
	
	\[
	T_h L_g \cdot X_h = X_{gh}, \quad \forall g, h \in G
	\]
	
	\begin{figure}[H]
		\centering
		\begin{tikzpicture}[scale=4, every node/.style={font=\small}]
			% Draw the object G, with boundary
			\draw[thick, blue!30!black] (0.2,0.2) .. controls (0,1) and (0.8,2.5) .. (2,2.2) 
			.. controls (2.5,1.8) and (2.3,0.8) .. (1.8,0.3) 
			.. controls (1.5,0.1) and (1,0.1) .. (0.2,0.2);
			\node at (-0.05,0.15) {$G$};
			
			% Draw points for g1, g2, h, gh
			\fill (0.7,0.55) circle (0.8pt);
			\node[left] at (0.65,0.55) {$g_1$};
			\fill (0.6,1.2) circle (0.8pt);
			\node[left] at (0.57,1.2) {$g_2$};
			\fill (1.15,1.1) circle (0.8pt);
			\node[below] at (1.2,1.08) {$h$};
			
			\fill (1.65,1.85) circle (0.8pt);
			\node[right] at (1.7,1.85) {$gh$};
			
			% Red arrows: left-invariant vector fields
			\draw[->,red,thick] (0.7,0.55) -- +(0.25,0.28) node[right] {$X_{g_1}$};
			\draw[->,red,thick] (0.6,1.2) -- +(0.45,0.22) node[right] {$X_{g_2}$};
			\draw[->,red,thick] (1.15,1.1) -- +(0.32,0.42) node[right] {$X_{h}$};
			\draw[->,red,thick] (1.65,1.85) -- +(0.22,0.15) node[right] {$X_{gh}$};
			
			% Blue arrow: pushforward
			\draw[->,blue,thick] (1.65,1.85) -- +(0.26,-0.08) 
			node[below right] {$T_h L_g X_h$};
		\end{tikzpicture}
	\end{figure}
	
	On note $\mathfrak{X}_L(G) \equiv$ champs inv. à gauche.
	
	\vspace{1em}
	
	\noindent
	\textbf{Prop}: $\mathfrak{X}_L(G)$ est une sous-algèbre de Lie de $\mathfrak{X}(G)$.
	
	\underline{Def.} \\
	$\mathfrak{X}_L(G) =$ algèbre de Lie de $G$.
	
	\vspace{1em}
	
	Soit $\xi \in T_e G$. On définit \\
	(uniquement) $X^\xi \in \mathfrak{X}_L(G)$ :
	
	\[
	(X^\xi)_g := T_e L_g \cdot \xi
	\]
	
	
	On montre que $X^\xi \in \mathfrak{X}_L(G)$. \\
	En effet, $\forall g, h \in G$,
\begin{align*}
	(X^\xi)_{gh} &= T_e L_{gh} \, \xi \\
	\phantom{(X^\xi)_{gh}} &= T_e(L_g \circ L_h) \, \xi \\
	\phantom{(X^\xi)_{gh}} &  = T_h L_g \circ T_e L_h \, \xi = T_h L_g (X^\xi_h) \qed \\
\end{align*}

\textbf{Prop.} L'application
\[
T_e G \longrightarrow \mathfrak{X}_L(G)
\]
\[
\xi \longmapsto X^\xi
\]
est un isomorphisme d'espaces vectoriels (d'e.v.).

\vspace{1em}

On peut dès lors transporter le crochet de Lie de $\mathfrak{X}_L(G)$ sur $T_e G$.

Pour tous $\xi, \eta \in T_e G$,
\[
[\xi, \eta]_L := [X^\xi, X^\eta] \big|_e
\]

Ainsi, $(T_e G, [\, , \, ]_L)$ est une algèbre de Lie isomorphe à l'algèbre de Lie $\mathfrak{X}_L(G)$ du groupe de Lie $G$.

\section{08.07.2025}

\textbf{Exo (Algèbre du Lie $\mathbb{H}_3$)}
\[
\mathbb{H}_3 = \mathbb{R}^3, \quad (a,b,c), (x,y,z) = (a+x, b+y, c+z+ay)
\]
\[
e = (0,0,0)
\]
\[
\mathfrak{h} = \operatorname{Lie}(\mathbb{H}_3) = \mathbb{R}^3 = \mathrm{vect}\{\xi_1, \xi_2, \xi_3\}
\]
\[
S_1 \; \xi = (\xi^1, \xi^2, \xi^3) \in \mathbb{R}^3, \quad \text{et } g=(x,y,z)
\]
\[
\text{on a :} \quad X|_g = T_e L_g \xi = 
\begin{pmatrix}
	1 & 0 & 0 \\
	0 & 1 & 0 \\
	0 & x & 1
\end{pmatrix}
\begin{pmatrix}
	\xi^1 \\ \xi^2 \\ \xi^3
\end{pmatrix}
=
\begin{pmatrix}
	\xi^1 \\ \xi^2 \\ x\xi^2 + \xi^3
\end{pmatrix}
\]

\begin{align*}
	L_g(h) &= g \cdot h\\
	&= (x, y, z)(u, v, w)\\
	&= (x + u, y + v, z + w + x v)
\end{align*}


\begin{align*}
	[H, P] &= HP - PH\\
	&= \frac{\partial}{\partial x}\left(\frac{\partial}{\partial y} + x \frac{\partial}{\partial z}\right)
	- \left(\frac{\partial}{\partial y} + x \frac{\partial}{\partial z}\right)\frac{\partial}{\partial x}\\
	&= \frac{\partial}{\partial z} = Q
\end{align*}
\[
\mathfrak{h}_3 = \mathrm{vect}\{\xi_1, \xi_2, \xi_3\} \quad \text{avec} \quad [\xi_1, \xi_2] = \xi_3.
\]
\[
\underline{\text{Ex:}} \quad \text{Déterminer } af(\mathbb{R}) = \mathrm{Lie}(Af(\mathbb{R}))
\]

\begin{align*}
	X^{\xi}_{(x_1, y_1, z_1)} &= \xi^1 \frac{\partial}{\partial x} + \xi^2 \frac{\partial}{\partial y}
	+ (x \xi^2 + \xi^3) \frac{\partial}{\partial z} \\
	&= \xi^1 \frac{\partial}{\partial x} + \xi^2\left(\frac{\partial}{\partial y} + x \frac{\partial}{\partial z}\right)
	+ \xi^3 \frac{\partial}{\partial z}
\end{align*}

Le champ $X^\xi$ s'écrit :
\begin{align*}
	X &= \xi^1 \frac{\partial}{\partial x} + \xi^2 \frac{\partial}{\partial y} + x \xi^2 \frac{\partial}{\partial z}
	+ \xi^3 \frac{\partial}{\partial z}
\end{align*}

Posons $H = \frac{\partial}{\partial x}$, $P = \frac{\partial}{\partial y} + x \frac{\partial}{\partial z}$

on a:
\[
Q = \frac{\partial}{\partial z}
\]

\section*{III. Application exponentielle}

Soit \( G \) un groupe de Lie, \\
\(\mathfrak{g} = T_e G\) son algèbre de Lie. \\

Si \( \xi \in \mathfrak{g} \), \( X \in \mathcal{X}_L(G) \) \\

On note \( \gamma_X : \mathbb{R} \longrightarrow G \) \\
la courbe intégrale de \( X \).

\[
\left\{
\begin{aligned}
	&\dot{\gamma}(t) = X^{\xi}_{\gamma(t)} \\
	&\gamma(0) = e
\end{aligned}
\right.
\]

\textbf{Déf.} \quad 
\[
\exp : \mathfrak{g} \longrightarrow G
\]
\[
\xi \longmapsto \gamma_\xi(1)
\]

\textbf{Prop.} \quad \(\exp\) est un difféo local (d'inv. réciproque).

\[
T_0 \exp = \mathrm{id}_{\mathfrak{g}}
\]


\textbf{Ex:} Déterminer \(\exp_{\mathcal{G}}\) dans les cas suivants: 
\begin{itemize}
	\item[1/] \(\mathcal{G} = (\mathbb{R}, +)\)
	\item[1' bis:] \(\mathcal{G} = (\mathbb{R}^*, \times)\)
	\item[2/] \(\mathbb{H}_3\), \(\operatorname{Aff}(\mathbb{R})\)
\end{itemize}


\textbf{Thm.} Soit \( G \) un groupe de Lie, \( H \) un sous-groupe de Lie.
\[
\mathfrak{h} = \left\{ \xi \in \mathfrak{g} \mid \exp(t \xi) \in H,\, \forall t \in \mathbb{R} \right\}
\]

\textbf{Ex.} Déterminer les algèbres de Lie de \( SL(n) \), \( SO(n) \), \( O(n) \), tous sous-groupes de Lie de \( GL(n, \mathbb{R}) \).

\section*{IV - 2\`eme Thm de Lie}

Étant donné \( G \), on sait lui associer une algèbre de Lie \(\mathfrak{g} = T_e G \simeq \mathcal{X}(\mathfrak{g})\). \\

\underline{Question.} Étant donnée une algèbre de Lie \( \mathfrak{A} \), \\
existe-t-il \( G \) tel que \( \mathfrak{A} = \operatorname{Lie}(G) \) ?

\textbf{Thm. (3ème de Lie)} \\
Soit $A$ une algèbre de Lie de dimension finie. Il existe un groupe de Lie $G$ d'algèbre de Lie isomorphe à $A$.

\[
\text{Quid de l'unicité?}
\]
\[
(\mathbb{R}, +) \text{ et } S^1 \text{ sont des groupes de Lie de même algèbre de Lie } \mathbb{R}.
\]

Mais ces deux groupes de Lie ne sont pas isomorphes.

\bigskip

\textbf{Thm (Lie-Cartan)} \\
Soit $\mathfrak{g}$ une algèbre de Lie de dimension finie. Il existe, à isomorphisme près, un unique groupe de Lie connexe et simplement connexe $\tilde{G}$ d'algèbre de Lie $\mathfrak{g}$.

Tout autre groupe de Lie connexe $G$ d'algèbre de Lie $\mathfrak{g}$ est de la forme $G = \tilde{G} / \Gamma$,\\
où $\Gamma$ est un sous-groupe discret du centre de $\tilde{G}$.

\bigskip

\textbf{Ex.} $\mathfrak{g} = \mathfrak{h}_3 = $ Heisenberg

\begin{itemize}
	\item $H_3 = \mathbb{R}^3$ \hspace{0.5cm} (Heisenberg group)
	\item $\mathcal{Z}(H_3) = \left\{
	\begin{pmatrix}
		1 & 0 & t \\
		0 & 1 & 0 \\
		0 & 0 & 1
	\end{pmatrix}
	\,\Bigg|\,
	t \in \mathbb{R}
	\right\}\simeq \mathbb{R}$
\end{itemize}

Des sous-groupes discrets sont $\Gamma_k \simeq k\mathbb{Z}$

\medskip

Les groupes connexes correspondants sont:

\[
G_k = H_3 / \Gamma_k
\]

On va prendre $k = 1$:

\[
G_1 = H_3 / (\mathbb{R}^2 \times \mathbb{Z})
\]

On obtient :

\[
G_1 = \mathbb{R}^2 \times (\mathbb{R}/\mathbb{Z}) = \mathbb{R}^2 \times S^1
\]

\textbf{Exercice :} Écrire le produit sur $G_1$.
\end{document}