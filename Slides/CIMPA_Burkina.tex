\documentclass[10pt]{beamer}
\usepackage{tikz}
\usetheme[progressbar=frametitle]{metropolis}
\usepackage{appendixnumberbeamer}
\usepackage{lipsum}
\usepackage{amsmath}
\usepackage{amssymb}
\usepackage{booktabs}
\usepackage[scale=2]{ccicons}
\usepackage{pgfplots}
\usepgfplotslibrary{dateplot}
\usepackage{dirtytalk}
\usepackage{xspace}
\usepackage[export]{adjustbox}
\newcommand{\themename}{\textbf{\textsc{metropolis}}\xspace}
\definecolor{mpigreen}{HTML}{007977}
\setbeamercolor{frametitle}{bg=mpigreen}
\usepackage{subfig}
\definecolor{mpigreen}{HTML}{007977}
\setbeamercolor{frametitle}{bg=mpigreen}
\usepackage{array}
\usepackage[most]{tcolorbox}
\usepackage{listings}
\usepackage{xcolor} % For coloring

% Set python style for listings
\definecolor{codebg}{rgb}{0.97,0.97,0.97}
\definecolor{codegray}{rgb}{0.5,0.5,0.5}
\definecolor{codepurple}{rgb}{0.58,0,0.82}

\lstdefinestyle{mypython}{
language=Python,
backgroundcolor=\color{codebg},
commentstyle=\color{codegray}\ttfamily\footnotesize,
keywordstyle=\color{blue}\bfseries,
numberstyle=\tiny\color{codegray},
stringstyle=\color{codepurple},
basicstyle=\ttfamily\footnotesize,
breaklines=true,
numbers=left,
numbersep=6pt,
frame=single,
rulecolor=\color{gray!60},
columns=fullflexible,
showstringspaces=false,
}

% --- Define colors for the shell (customizable) ---
\definecolor{shellbg}{rgb}{0.05,0.07,0.07}
\definecolor{shellprompt}{rgb}{0.3,0.9,0.4}
\definecolor{shelltext}{rgb}{0.9,0.9,0.9}

% --- Define a custom shell style, with prompt coloring ---
\lstdefinelanguage{bash}{
sensitive=true,
alsoletter={-,/},
morekeywords={sudo,pwd,cd,mkdir,rm,ls,cp,mv,find,echo,export, Python},
}
\newtcolorbox[auto counter,number within=section
]{debox}[1][]{enhanced jigsaw,
colback=green!0!white,
coltext={black},
colframe={black},
coltitle={black},
boxrule=0pt,
frame hidden,
borderline west={1mm}{0mm}{white!50!red},
arc=0mm,
auto outer arc,
boxsep=5pt,
left=4pt,
breakable,
right=4pt,
bottom=0pt,
top=0pt,
before skip=3mm,
after skip=3mm,
%label={def:\thetcbcounter},
#1}

\newcommand{\gd}[2]{\begin{debox}[label=#1]#2\end{debox}}

\lstdefinestyle{advancedshell}{
language=bash,
backgroundcolor=\color{shellbg},
basicstyle=\ttfamily\footnotesize\color{shelltext},
frame=single,
rulecolor=\color{shellprompt},
breaklines=true,
showstringspaces=false,
columns=fullflexible,
numbers=none,
aboveskip=0.8em,
belowskip=0.8em,
literate={\$}{{\textcolor{shellprompt}{\$}}}{1}
% You can add more, such as user/host prompt: {username@host:~\$}{\textcolor{shellprompt}{username@host:~\$}}{15}
}
\title{Calcul Mathématique avec \texttt{Python}/\texttt{SageMath}}

\author{ \textbf{Cyrille Merleau Nono Saha} } 
% \\Supervised by: Dr. Matteo Smerlak}
\vspace{7cm}
\institute{
\textbf{CIMPA TRAP 2025} \\ \\ 
University of Leipzig/ScaDS.AI\\
Max Planck Institute for Mathematics in the Sciences\\
Lancaster Leipzig University}

\titlegraphic{

\begin{tikzpicture}[overlay,remember picture]
\node[right=0.6cm] at (current page.150){
\includegraphics[width=3cm]{images/MPI-Logo.pdf}
};
\end{tikzpicture}

\begin{tikzpicture}[overlay,remember picture]
\node[left=0.6cm] at (current page.30){
\includegraphics[width=3cm]{images/Scads-logo.png}
};
\end{tikzpicture}
}

\begin{document}
%\section{Biological context }
\maketitle

\begin{frame}{Plan du cours}
%\scriptsize
\setbeamertemplate{section in toc}[sections numbered]
\tableofcontents[hideallsubsections]
%\thispagestyle{empty}
\end{frame}

\section[Team and resources]{\textbf{Team and resources}} 

\begin{frame}{Team and resources}
\begin{block}{Instructors}
\begin{itemize}
	\item Nono Saha Cyrille Merleau
	\item The participants (You)
\end{itemize}
\end{block}
\begin{block}{Course resources (s)}
\begin{itemize}
	\item GitHub
	\item Readings and link to the course tools
	\item No specific material: A reading list is given on the references page. 
\end{itemize}
\end{block}


\begin{block}{Course Timetable (s)}
\begin{itemize}
	\item Lectures: 01.07, 04.07, 08.07, and 10.07
\end{itemize}
\end{block}
\end{frame}

%\section{\textbf{Organisation du cours}}
%
%\begin{frame}{Course Organisation}
%	\begin{block}{Objectifs}
%		\begin{itemize}
%			\item Setting up our programming environment
%			\item Introduction to \texttt{Python} Programming
%			\item Object Oriented Programming with \texttt{Python}
%			\item Introduction to symbolic computation with \textbf{Sympy}
%			\item Introduction to Group Theory with \texttt{SageMath}/\texttt{Python}
%		\end{itemize}
%	\end{block}
%\end{frame}

%
%\section{\textbf{Before next Wednesday...}}
%\begin{frame}{Before next Wednesday...}
%	
%	\begin{block}{Introductory survey}
%		\vspace*{0.2cm}
%		Please follow this link and answer the questions to get to know each other.  \small \url{https://forms.office.com/e/Vm3memrUjz?origin=lprLinkk}
%	\end{block}
%	
%	\begin{block}{First home work}
%		\begin{enumerate}
%			\item Create a GitLab account if you do not have one yet 
%			\item Ensure your programming environment is set up correctly.
%			\item Complete the course survey (the link is given above).
%			\item If 1, 2 and 3 are done, create your first GitLab repo containing a README.md file. It will briefly describe the steps you took to set up your environment. e.g. the OS, the command used, the problems encountered, how you solved them, etc...
%		\end{enumerate}
%	\end{block}
%	
%\end{frame}

\section[Programming environment]{\textbf{Programming environment}} 

\begin{frame}[fragile]{Setting your environment up}

\begin{itemize}
\item \href{https://docs.conda.io/projects/miniconda/en/latest/}{Conda} with a specific python version
\begin{lstlisting}[style=advancedshell]
$ conda --version 
\end{lstlisting}
\item Check your package manager 
\begin{lstlisting}[style=advancedshell, numbers=none]
$ pip --version or pip3 --version 
\end{lstlisting}

\item Create your conda environment
\begin{lstlisting}[style=advancedshell]
$conda create --name CIMPA sage python=3.11
$conda activate CIMPA
\end{lstlisting}
\item Run sage on your 
\begin{lstlisting}[style=advancedshell, numbers=none]
$ sage
\end{lstlisting}
\end{itemize}

\end{frame}

\begin{frame}{Introduction to Git}
\begin{block}{What is Git?}
\begin{itemize}
	\item Modern version control system
	\item Mature, actively maintained, open-source project (Linus Torvalds, 2005)
	\item Works well on a wide range of operating systems and IDEs
\end{itemize}
\end{block}
\begin{block}{Main characteristics}
\begin{itemize}
	\item Distributed VCS \emph{vs} CVC 
	\item Performance: Algorithms, File content/names
	\item Security: Top priority and SHA1 hashing algorithm  for commits, content, file-folder
	\item Flexibility: small and large projects, many OS, branches and tags
\end{itemize}
\end{block}

\end{frame}

\begin{frame}{Essential git commands} 
\begin{block}{Init and clone directories}
\begin{itemize}
	\item \textbf{git init}: creates an empty Git repository 
	\item \textbf{git clone}: clone a repository into a new directory
	\item More importantly, learn to use the "manual"
\end{itemize}
\end{block}
\begin{block}{Basic commands}
\begin{itemize}
	\item \textbf{git pull}: get recent updates from the remote to the local branch
	\item \textbf{git add } <file or folder>: add file contents to the index
	\item \textbf{git commit }-m <some comments>: record changes to the repository
	\item \textbf{git push} <origin> <branch>: update remote refs along with associated objects
\end{itemize}
\end{block}
\begin{block}{More commands}
\begin{itemize}
	\item \textbf{git merge},  \textbf{git fetch}, \textbf{git status}, \textbf{git log}
	\item  \href{https://git-scm.com/docs}{More on Git Document}
\end{itemize}
\end{block}
\end{frame}


\begin{frame}{Introduction to \texttt{Python}}

\begin{block}{What is \texttt{Python}?}
\begin{itemize}
	\item A programming language that boasts ease of use
	\item Dynamic typing and garbage-collected language
	\item Batteries included (\small \url{pypi.python.org})
\end{itemize}
\end{block}

\begin{block}{Several advantages}
\begin{itemize}
	\item Code readability with the use of significant indentation via the off-side rule
	\item High level and for general purposes
	\item Structured, functional and OO-programming 
\end{itemize}
\end{block}

\begin{block}{Important!!!!!}
\begin{itemize}
	\item  Network sockets, database handles, windows, and file descriptors are not included in the garbage-collection
	\item Need of other methods (e.g. destructors)
\end{itemize}
\end{block}
\end{frame}


\begin{frame}{Variables: basic types}
\begin{block}{What is a variable?}

\begin{itemize}
	\item A value stored in computer memory
	\item It should have a name and store the corresponding value
	\item Use a combination of alphanumeric characters and the underscore character for names
	\item Convention recommends lowercase characters with words separated by an underscore for readability
\end{itemize}
\end{block}

\begin{block}{Type system:}

\begin{itemize}
	\item \textbf{Boolean} (or \textbf{bool}): e.g.  True, False 
	\item  \textbf{Float}: e.g. 1., 1.0, 2.4
	\item \textbf{Integer} (or \textbf{int}): e.g. 1, 34
	\item \textbf{Complex Number} (or \textbf{complex}): e.g. 1+1j , 1+0j, 4j, etc..
\end{itemize}
\end{block}
\end{frame}

\begin{frame}[fragile]{Variable: dynamic types}  % add fragile to allow using lstlisting in frame
Therefore, most basic operations will coerce a variable to a consistent type suitable for the operation.

\begin{block}{Boolean operations and comparisons}
\vspace{-0.5cm}
\begin{table}
	\centering
\begin{tabular}{p{5cm}p{5cm}p{5cm}}
\begin{lstlisting}[style=mypython]
1 and True
## True	
0 or 1
## 1 
not 0
## True	
not (0+0j)
## True	
not (0+1j)
## False						
\end{lstlisting}&
\begin{lstlisting}[style=mypython, numbers=none]
5. > 1
## True
5. ==  5 
## True
1 > True
## False
(1+0j) == 1
## True
'abc' < "ABC"
## False 
\end{lstlisting}
\end{tabular}
\end{table}
\end{block}
\end{frame}
\begin{frame}[fragile]{Variables: basic mathematical operations}

\begin{block}{Math operations}
\vspace{-0.85cm}
\begin{table}
	\centering		
\begin{tabular}{p{5cm}p{5cm}p{5cm}}
\begin{lstlisting}[style=mypython]
1 + 5
## 6
1 + 5.
## 6.
1 * 5.
## 5.0
True * 5
## 5
(1 + 0j) - (1 + 1j)	
## -1j 			
\end{lstlisting}&
\begin{lstlisting}[style=mypython, numbers=none]
5 / 1.
## 5.0
5 / 2
## 2.5
5 // 2 
## 2
5 % 2 
## 1
7 ** 2
## 49 
\end{lstlisting}
\end{tabular}
\end{table}
\vspace{-0.9cm}
Now, what if I do? 
\begin{lstlisting}[style=mypython]
"abc" + 5  	## Error? What type of error? Why? 
"abc" + str(5) ## Does that correct the prev error?
"abc" ** 2 # What about this? 
"abc" * 3  # And this?
\end{lstlisting}
\end{block}

\end{frame}

\begin{frame}[fragile]{Variables: casting and assignment}
\begin{block}{Casting using type functions: e.g. float(), int(), etc..}
\vspace*{-0.5cm}
\begin{tabular}{ p{4cm} p{4cm} }
	\centering
	\begin{lstlisting}[style=mypython]
float ("0.5")
## 0.5
float(True)	
## 1.0
int(1.1)
## 1
int("2")
## 2
	\end{lstlisting} & 
	\begin{lstlisting}[style=mypython, numbers=none]
bool(0)
## False
bool("hello")
## ??
str(3.14159) 
## "3.14159"
str(True) 
## "True"
	\end{lstlisting}   
\end{tabular}

\vspace*{-0.7cm}
Now, what if I do? 
\begin{lstlisting}[style=mypython]
int("2.1")
## Error? What type of error? Why? 
\end{lstlisting}
\vspace*{-0.15cm}
Variable assignment
\begin{lstlisting}[style=mypython]
x = 100 ## Assign a value of 100 to a variable named x
a = b = 5 ## Assign a value of 5 to both variables a and b
\end{lstlisting}
\end{block}
\end{frame}


\begin{frame}[fragile]{Variables: special values }
No missing values and non-finite floating point values are available. There is a None type similar to  NULL in \texttt{R}, \texttt{Java}, \texttt{JavaScript}.
\begin{lstlisting}[style=mypython]
1 / 0
## Error in py_call_impl(callable, dots$args, 
dots$keywords):
ZeroDivisionError: division by zero
## Detailed traceback:
## File "<string>", line 1, in <module>

1. / 0 # qu'en est il de ceci? 

float("nan")
## nan
float("-inf")
## -inf, we can do 5 > float("inf")
\end{lstlisting}
\end{frame}

\begin{frame}[fragile]{Variables: string literals}
\vspace*{-0.1cm}
Strings can be defined using a couple of different ways,
\begin{lstlisting}[style=mypython]
'allows embedded "double" quotes'
##' allows embedded "double" quotes'
\end{lstlisting}

\begin{lstlisting}[style=mypython]
"allows embedded 'single' quotes"
## "allows embedded 'single' quotes"
\end{lstlisting}
\vspace*{-0.2cm}
Strings can also be triple quoted, using single or double quotes, which allows the string to span multiple lines.
\begin{lstlisting}[style=mypython]
"""line one line two line three""" 
## 'line one\nline two\nline three'
\end{lstlisting}
\vspace*{-0.1cm}
Several methods are possible: 
\begin{lstlisting}[style=mypython]
x = "Hello wolrd! 1234"
x.find("!"), x.isalnum(), x.title(), x.swapcase(),
x.split()
\end{lstlisting}
\end{frame}


\begin{frame}[fragile]{Variables: sequence types}
\begin{block} {Lists in \texttt{Python}}
\vspace*{0.1cm}
Python lists are heterogeneous, ordered, mutable containers of objects.

\begin{lstlisting}[style=mypython]
x = [0,1,1,0]; x
## [0, 1, 1, 0], we can use subsetting with x[start:stop:step]

[0, True, "abc"]
## [0, True, 'abc'] mutate an element, x[-1] = 2

[0, [1,2], [3,[4]]]
## [0, [1, 2], [3, [4]]], can we assign?

x = [0,1,1,0]
type(x)
## <class 'list'> we can sort with x.sort()

y = [0, True, "abc"]
type(y)
## <class 'list'> is y.sort() still work?
\end{lstlisting}
\end{block}

\end{frame}

\begin{frame}[fragile]{Variables: sequence types}
\begin{block} {Unpacking lists in \texttt{Python}}
\vspace*{0.2cm}

Unpacking into multiple variables when doing "assignment",


\begin{lstlisting}[style=mypython]
x, y = [1,2]
x
## 1 similarly we can do x, y = [[0,1], [2, 3]]
y
## 2
x, y = [1, [2, 3]]
x
## 1 or something like (x1,y1), (x2,y2)=[[0,1], [2, 3]]
y
## [2, 3]
\end{lstlisting}
\vspace*{-0.2cm}
Extended unpacking:
\begin{lstlisting}[style=mypython]
x, *y = [1,2,3] ## what about this x, y = [1,2,3]? 
y ## [2, 3] what about *x, y = [1,2,3]?
\end{lstlisting}
\end{block}
\end{frame}

\begin{frame}[fragile]{Variables: sequence types}
\begin{block} {Tuples in \texttt{Python}}
\vspace*{0.1cm}
\texttt{Python} tuples are heterogenous, ordered, immutable (or non-mutable) containers of values.

\begin{lstlisting}[style=mypython]
(1,2,3)
## (1, 2, 3)

(1,True," abc")
## (1, True, 'abc')

(1,(2,3))
## (1, (2, 3))

x = (1,2,3)
x[2] = 5 ## What will happen here? 
\end{lstlisting}
\end{block}

\end{frame}


\begin{frame}[fragile]{Variable: ranges as a sequence type}
These are the last sequence types and are somewhat special - ranges are homogeneous, ordered, and immutable "containers" of integers.

\textbf{Examples}: 
\vspace{0.2cm}
\begin{lstlisting}[style=mypython]
range(10) ## range(0, 10)

range(0,10) ## range(0, 10)

range(0,10,2) ## range(0, 10, 2)

range(10,0,-1) ## range(10, 0, -1)

list(range(10)) ## [0, 1, 2, 3, 4, 5, 6, 7, 8, 9]. )
\end{lstlisting}

\textbf{Remarque}: What about this \texttt{list(range(10,0,-1))}?
\end{frame}

\begin{frame}{Homework}
\begin{block}{Programming like a hipster! }
\vspace*{0.3cm}
\begin{itemize}
	\item Write a program that computes a square root of any given integer. NB: making use of no Python library.
	\item Given a list, write a program that returns an ascendent sorted list.
\end{itemize}
\end{block}
\end{frame}

\section{\textbf{An introduction to OOP}}

%\section{Class, interface, abstract class and object}
\begin{frame}[fragile]{Basic syntax}
These are the basic components of Python's object-oriented system.

\begin{lstlisting}[style=mypython]
class Rectangle:
	"""An abstract representation of a rectangle"""
	# Attributes
	p1 = (0,0)
	p2 = (1,2)

	# Methods
	def area(self):
		return abs(self.p1[0] - self.p2[0]) 
		*abs(self.p1[1] - self.p2[1])
	
	# Setters
	def setP1(self, p1):
		self.p1 = p1
\end{lstlisting}
\end{frame}

%\section{\textbf{Interfaces and abstract classes}}
\begin{frame}[fragile]{Interfaces and abstract classes}
\begin{block}{Two important points}
\begin{itemize}
	\item Interfaces are classes that contain methods without implementations
	\item Abstract classes are classes with at least one method without implementation
\end{itemize}
\end{block}

\begin{block}{Example}

\vspace{0.5cm}
\begin{lstlisting}[style=mypython]
class AbstractRectangle(abc.ABC):   
	def __init__(self, p1=(0,0), p2=(1,2)):         
		self.p1 = p1
		self.p2 = p2
	
	@abc.abstractmethod
	def area(self): 
		pass 
\end{lstlisting}
\end{block}
\end{frame}

\begin{frame}[fragile]{Objects}
\begin{block}{What is an object?}
\begin{itemize}
	\item Objects are instances of a class
	\item Objects can also represent different states of a class
	\item Objects are coherent entities that store data and the code (or instructions) working on that data. 
\end{itemize}
\end{block}

\begin{block}{Example}
\vspace{0.5cm}
\begin{lstlisting}[style=mypython]
x = Rectangle()
x.area()
\end{lstlisting}

\begin{lstlisting}[style=mypython]
y = Rectangle(p2= (5,4))
y.area()
\end{lstlisting}
\end{block}
\end{frame}


\begin{frame}[fragile]{Class attributes}
We can examine all of a class's methods and attributes using \texttt{dir()},
\begin{lstlisting}[style=mypython]
dir(Rectangle)
['__class__','__delattr__','__dict__','__dir__',
'__doc__','__eq__','__format__','__ge__',
'__getattribute__','__gt__','__hash__',
'__init__','__init_subclass__','__iter__','__le__',
'__lt__','__module__','__ne__','__new__',
'__reduce__','__reduce_ex__','__repr__',
'__setattr__','__sizeof__','__str__',
'__subclasshook__','__weakref__','area']	
\end{lstlisting}
Where did p1 and p2 go?
\begin{lstlisting}[style=mypython]
dir(Rectangle())
['__class__','__delattr__','__dict__','__dir__'...
\end{lstlisting}
\end{frame}

\begin{frame}[fragile]{Instantiation (constructors)}
When instantiating a class (e.g. \texttt{Rectangle()}) we invoke the \texttt{\_\_init\_\_()} method if it is present in the classes' definition.

\vspace{1cm}

\begin{lstlisting}[style=mypython]
class Rectangle:
	"""An abstract representation of a rectangle"""
	# Constructor
	def __init__(self, p1 = (0,0), p2 = (1,1)):
		self.p1 = p1
		self.p2 = p2
	
	# Methods
	def area(self):
		return ((self.p1[0] - self.p2[0])
		*(self.p1[1] - self.p2[1]))
\end{lstlisting}
\end{frame}

\begin{frame}[fragile]{Method chaining}
We will see several objects (e.g., \texttt{exp(-x).diff().diff()}) that allow for method chaining to construct a pipeline of operations. We can achieve the same effect by having our class methods return 'self'.\\
\vspace{1cm}

\begin{lstlisting}[style=mypython]
class Rectangle:
	"""An abstract representation of a rectangle"""
	# Constructor
	def __init__(self, p1 = (0,0), p2 = (1,1)):
		self.p1 = p1
		self.p2 = p2
	# Methods
	def area(self):
		return ((self.p1[0] - self.p2[0])
		*(self.p1[1] - self.p2[1]))
\end{lstlisting}

\end{frame}

\begin{frame}[fragile]{Object string formating}
All class objects have a default print method/string conversion method, but the default behaviour is not very useful,
\begin{lstlisting}[style=mypython]
print(Rectangle())
## <__main__.rect object at 0x290aa1a60>

str(Rectangle())
## '<__main__.rect object at 0x290aa1ca0>'
\end{lstlisting}
Both of the above are handled by the \texttt{\_\_str\_\_()} method, which is implicitly created for our class - we can override this,
\begin{lstlisting}[style=mypython]
def rect_str(self):
	return f"Rectangle[{self.p1}, {self.p2}]
	area={self.area()}"

Rectangle.__str__ = rect_str
\end{lstlisting}
\end{frame}

\begin{frame}[fragile]{Class representation}
There is another special method responsible for printing the object (see \texttt{Rectangle()} above), called \texttt{\_\_repr\_\_()}, which is used to print the class representation. If possible, this is intended to be a valid Python expression that can recreate the object.
\begin{lstlisting}[style=mypython]
def rect_repr(self):
	return f"Rectangle({self.p1}, {self.p2})"

rect.__repr__ = rect_repr
\end{lstlisting}

\begin{lstlisting}[style=mypython]
Rectangle()
## Rectangle((0, 0), (1, 1))
\end{lstlisting}

\begin{lstlisting}[style=mypython]
repr(Rectangle())
## Rectangle((0, 0), (1, 1))
\end{lstlisting}
\end{frame}
\begin{frame}[fragile]{OOP:  Inheritance}
Part of the object-oriented system is that classes can inherit from other classes, meaning they gain access to all of their parent's attributes and methods. It models a \textbf{Is a} relationship.

In an inheritance relationship:

\begin{itemize}
\item Classes inherited from another are derived classes, subclasses, or subtypes.
\item Classes from which other classes are derived are called base classes or superclasses.
\item A derived class is said to derive, inherit, or extend a base class.
\end{itemize}
\begin{lstlisting}[style=mypython]
class Square(Rectangle): 
	pass 
\end{lstlisting}
\begin{lstlisting}[style=mypython]
Square()
## Rectangle((0, 0), (1, 1))
\end{lstlisting}
\end{frame}

\begin{frame}[fragile]{OOP: Multiple inheritance}
A class can be derived from more than one superclass in Python. This is called multiple inheritance.
\begin{lstlisting}[style=mypython]
class Worm:
	def __init__ (self, name): 
		self.name = name
	def eat(self): 
		print(self.name +" swallows")

class Fly:
	def __init__ (self, name): 
		self.name = name
	def eat(self): 
		print(self.name +" is nibbling..")

class ButterFly(Worm, Fly): 
	pass
\end{lstlisting}


\end{frame}

\begin{frame}[fragile]{OOP Inheritance: overriding methods}
\begin{lstlisting}[style=mypython]
class Square(Rectangle):
	def __init__(self, p1=(0,0), l=1):
		assert isinstance(l, (float, int)), "numnber, please"
		p2 = (p1[0]+l, p1[1]+l)
		self.l  = l
		super().__init__(p1, p2)

	def setP1(self, p1):
		self.p1 = p1
		self.p2 = (self.p1[0]+self.l, self.p1[1]+self.l)
		return self
	def setP2(self, p2):
		raise RuntimeError("Squares take l not p2")
\end{lstlisting}
\end{frame}

\begin{frame}[fragile]{OOP: Making an object iterable}
When using an object with a for loop, Python looks for the \texttt{\_\_iter\_\_()} method, which is expected to return an iterator object (e.g. iter() of a list, tuple, etc...).
\begin{lstlisting}[style=mypython]
class Rectangle:
""" An object representation of a rectangle"""
# Constructor
	def __init__(self, p1 = (0,0), p2 = (1,1)):
		self.p1 = p1; self.p2 = p2
	# Methods
	def area(self):
		return ((self.p1[0] - self.p2[0]) 
		*(self.p1[1] - self.p2[1]))
	def __iter__(self):
		return iter( [ self.p1, (self.p1[0],
					self.p2[1]), self.p2,(self.p2[0],
					self.p1[1])])
\end{lstlisting}
\end{frame}


\begin{frame}[fragile]{OOP: Composition in \texttt{Python}}
Composition is an OOP concept that models a relationship. In composition, a class known as a composite contains an object of another class, referred to as a component. In other words, a composite class has a component of another class.

\textbf{Example}: We already used it in the previous Example, but how? and where?

\begin{block}{Remarks}
\begin{itemize}
	\item Composition is more flexible than inheritance because it models a loosely coupled relationship
	\item Changes to a component class have minimal or no effects on the composite class
	\item Designs based on composition are more suitable for change
\end{itemize}
\end{block}
\end{frame}

\begin{frame}[fragile]{Example of composition in \texttt{Python}}
\begin{lstlisting}[style=mypython]
class Salary:
	def __init__(self, pay, bonus):
		self.pay = pay
		self.bonus = bonus
	def annual_salary(self):
		return (self.pay*12)+self.bonus

class EmployeeOne:
	def __init__(self, name, age, pay, bonus):
		self.name = name
		self.age = age
		self.obj_salary = Salary(pay, bonus)  # comp
	def total_sal(self):
		return self.obj_salary.annual_salary()
\end{lstlisting}
\end{frame}

\begin{frame}[fragile]{OOP: Aggregation in \texttt{Python}}
\textbf{Aggregation} is a concept in which an object of one class can own or access another independent object of another class. 

\begin{itemize}
\item It represents \textbf{Has-A}'s relationship.
\item It is a unidirectional association, i.e. a one-way relationship. For Example, a department can have students, but the opposite is not possible, and thus it is unidirectional.
\item In Aggregation, both entries can survive individually, which means ending one entity will not affect another.	
\end{itemize}

\end{frame}

\begin{frame}[fragile]{Example of aggregation in \texttt{Python}}
\begin{lstlisting}[style=mypython]
class Salary:
	def __init__(self, pay, bonus):
		self.pay = pay
		self.bonus = bonus
	
	def annual_salary(self):
		return (self.pay*12)+self.bonus
	
class EmployeeOne:
	def __init__(self, name, age, sal):
		self.name = name
		self.age = age
		self.agg_salary = sal   # Aggregation
	
	def total_sal(self):
		return self.agg_salary.annual_salary()
\end{lstlisting}
%salary = Salary(10000, 1500)
%emp = EmployeeOne('Geek', 25, salary)  
%print(emp.total_sal())
\end{frame}

\begin{frame}[fragile]{OOP: more relationships...}
\begin{itemize}
\item \textbf{Association}: which expresses a uses-a relationship. For Example, a student may be associated with a course. They will use the course. This relationship is common in database systems, where it is often represented by one-to-one, one-to-many, and many-to-many associations. 
\item \textbf{Delegation}: which models a \textbf{can-do} relationship, where an object hands a task over to another object, which takes care of executing the task.
\item \textbf{Dependency injection}: a design pattern you can use to achieve loose coupling between a class and its components. With this technique, you can provide an object's dependencies from the outside rather than inheriting or implementing them in the object itself.
\end{itemize}
\end{frame}

\begin{frame}[fragile]{Polymorphism in \texttt{Python}}
A set of classes implementing the same interface with specific behaviours for concrete classes is a great way to unlock Polymorphism.

\textbf{Polymorphism} is when you can use objects of different classes interchangeably because they share a standard interface. 

\begin{block}{Example}
\vspace{0.1cm}
Python strings, lists, and tuples are all sequence data types. This means that they implement an interface that's common to all sequences.
\end{block}

We can use them in similar ways. For Example, you can:

\begin{itemize}
\item Loop them because they provide the \texttt{.\_\_iter\_\_()} method
\item Items are accessed through the \texttt{\_\_getitem\_\_()} method
\item Determine their number of items because they include the \texttt{.\_\_len\_\_()} method

\end{itemize}

\end{frame}

\begin{frame}[fragile]{Wrapping up the OOP in \texttt{Python}}


\begin{enumerate}
\item Python classes and how to use them to make your code more reusable, modular, flexible, and maintainable
\item Classes are the building blocks of object-oriented programming in Python
\item With classes, you can solve complex problems by modelling real-world objects, their properties, and their behaviours
\item Classes provide an intuitive and human-friendly approach to complex programming problems, making your life more pleasant.
\item We can use special classes such as interfaces and abstract classes to unlock properties like Polymorphism in python
\item Classes can interact through associations, aggregations, composition, inheritance, dependency injection and delegation 
\end{enumerate}

\end{frame}

\begin{frame}{Exercise: Object-Oriented Programming in Python}

\textbf{Tasks:}
\begin{enumerate}
\item Define a Python class \texttt{Book} with attributes for \texttt{title}, \texttt{author}, and \texttt{year}.
\item Write an \texttt{\_\_init\_\_} method that initializes these attributes.
\item Add a method \texttt{description} that returns a string like \\
\texttt{"Title by Author (Year)"}.
\item Create an instance of your class for the book "The Hobbit" by J.R.R. Tolkien, published in 1937, and print its description.
\end{enumerate}

\vspace{2mm}
\textbf{Optional:} Extend the class with a method to check if the book is a classic (published before 1970).
\end{frame}

\begin{frame}[fragile]{Solution: Object-Oriented Programming in \texttt{Python}}
\begin{lstlisting}[style=mypython]
class Book:
	def __init__(self, title, author, year):
		self.title = title
		self.author = author
		self.year = year

	def description(self):
		return f"{self.title} by {self.author} ({self.year})"

	def is_classic(self):
		return self.year < 1970

# Create an instance
hobbit = Book("The Hobbit", "J.R.R. Tolkien", 1937)
print(hobbit.description())           # The Hobbit by J.R.R. Tolkien (1937)
print(hobbit.is_classic())            # True
\end{lstlisting}
\end{frame}

\begin{frame}{Exercise: Object-Oriented Programming — Authors and Books}

Define two Python classes representing books and authors.

\textbf{Tasks:}
\begin{enumerate}
\item Define a class \texttt{Author} with attributes: \texttt{firstname}, \texttt{lastname}, \texttt{affiliation}, and \texttt{address}.
\item Define a class \texttt{Book} with attributes: \texttt{title}, \texttt{year}, and \texttt{author} (where \texttt{author} is an object of class \texttt{Author}).
\item Write appropriate \texttt{\_\_init\_\_} methods for both classes.
\item Add a \texttt{description} method in \texttt{Book} that returns a string like:\\
\texttt{"Title (Year) by Firstname Lastname, Affiliation"}.
\item Create an instance of \texttt{Author} for ``J.R.R. Tolkien'' (affiliated with ``University of Oxford'', address: ``Oxford, UK'') and a \texttt{Book} instance for ``The Hobbit'' (1937, by Tolkien). Print the book description.
\end{enumerate}
\end{frame}

\begin{frame}[fragile]{Solution: Author and Book Classes in Python}
\begin{lstlisting}[style=mypython]
class Author:
def __init__(self, firstname, lastname, affiliation, address):
	self.firstname = firstname
	self.lastname = lastname
	self.affiliation = affiliation
	self.address = address
class Book:
	def __init__(self, title, year, author):
		self.title = title
		self.year = year
		self.author = author  # Author object

def description(self):
	return f"{self.title} ({self.year}) by {self.author.firstname} {self.author.lastname}, {self.author.affiliation}"

# Create Author and Book instances
tolkien = Author("J.R.R.", "Tolkien", "University of Oxford", "Oxford,UK")
hobbit = Book("The Hobbit", 1937, tolkien)
print(hobbit.description())
\end{lstlisting}
\end{frame}
\section{\textbf{\texttt{Sympy} and symbolic mathematics}}

\begin{frame}[fragile]{Sympy basics: importing a module/package in Python}
\begin{block} {What is \texttt{Sympy}?}
\vspace*{0.1cm}
\gd{impliciteintersection}{
SymPy is a Python library that enables symbolic mathematics, including calculus operations like differentiation and integration. It allows users to work with mathematical expressions, equations, and functions symbolically rather than just numerical computation}
\end{block}

\begin{block}{\textbf{How to install \texttt{Sympy}?}}
\vspace{0.2cm}
\begin{lstlisting}[style=mypython]
pip install sympy 
\end{lstlisting}
\end{block}

\begin{block}{\textbf{Example test}}
\vspace{0.2cm}
\begin{lstlisting}[style=mypython]
import sympy as sym
x, y = sym.symbols('x y')
\end{lstlisting}
\end{block}

\end{frame}

\begin{frame}[fragile]{Sympy basics: differentiation, integration, and limits.}
\begin{block} {Defferentiation}
\vspace*{0.1cm}
Use \texttt{sympy.diff(function, variable)} to find the derivative of a function with respect to a variable:
\begin{lstlisting}[style=mypython]
f = x**2 + 2*x + 1
df_dx = sym.diff(f, x)  # Derivative of f with respect to x
print(df_dx)  # Output: 2*x + 2
\end{lstlisting}
\end{block}	
\vspace{-0.4cm}
\begin{block}{\textbf{Integration}}
\vspace{0.1cm}
\begin{lstlisting}[style=mypython]
integral_f = sym.integrate(f, x) #Indefinite integral of f with respect to x
print(integral_f) #Output: x**3/3 + x**2 + x
\end{lstlisting}
\end{block}
\vspace{-0.4cm}
\begin{block}{\textbf{Limits}}
\vspace{0.1cm}
Use sympy.limit(function, variable, point) to evaluate the limit of a function as the variable approaches a specific point: 
\begin{lstlisting}[style=mypython]
limit_f = sym.limit((sym.sin(x) / x), x, 0)
print(limit_f) # Output: 1
\end{lstlisting}
\end{block}

\end{frame}

\begin{frame}[fragile]{Application to Physics (1): Uniformly Accelerated Motion }

The position of an object under constant acceleration:\vspace{2mm}
\[
x(t) = x_0 + v_0 t + \frac{1}{2} a t^2
\]

\vspace{2mm}
\textbf{Symbolic computation with \texttt{sympy}:}

\begin{lstlisting}[style=mypython]
from sympy import symbols, Eq, solve, N

# Define symbols
x, x0, v0, a, t = symbols('x x0 v0 a t')

# Kinematics equation
eq = Eq(x, x0 + v0*t + (1/2)*a*t**2)

# Given: x0 = 0, v0 = 5 (m/s), a = 2 (m/s^2), t = 3
subs = {x0: 0, v0: 5, a: 2, t: 3}
x_val = eq.subs(subs)

print("Position at t=3s:", N(x_val.rhs))  # Output: 21.0
\end{lstlisting}

\end{frame}

\begin{frame}[fragile]{Application to Physics (2): The Ideal Gas Law}

The ideal gas equation relates pressure, volume, temperature, and the number of moles:
\[
PV = nRT
\]
Suppose we wish to solve for the pressure $P$.

\vspace{0mm}
\textbf{Symbolic computation with \texttt{sympy}:}

\begin{lstlisting}[style=mypython]
P, V, n, R, T = symbols('P V n R T') # Define symbols

# Ideal gas law
eq = Eq(P*V, n*R*T)

# Solve for pressure
P_sol = solve(eq, P)[0]
print("P =", P_sol)

# Numerical example: V=10 L, n=2 mol, T=300 K, R=0.0821 L*atm/(mol*K)
values = {V: 10, n: 2, T: 300, R: 0.0821}
P_num = P_sol.subs(values)
print("Pressure:", P_num)        # Output: 4.926
\end{lstlisting}

\vspace{2mm}
\textbf{Result:} $P = \dfrac{nRT}{V}$, and for the given values, $P \approx 4.93$~atm.
\end{frame}

\begin{frame}[fragile]{Solving ODE Systems via Laplace Transform with \texttt{sympy}}
\vspace{-0.3cm}
\[
\dot{x} = 3x + 4y,\quad \dot{y} = -4x + 3y,\quad x(0) = 1,\, y(0) = 0
\]
%	\textbf{SymPy Solution (Laplace Transform):}
\begin{lstlisting}[style=mypython]
from sympy import symbols, Function, laplace_transform, inverse_laplace_transform, dsolve
from sympy.abc import t, s

# Define functions
x = Function('x')
y = Function('y')

# System as equations
eq1 = x(t).diff(t) - 3*x(t) - 4*y(t)
eq2 = y(t).diff(t) + 4*x(t) - 3*y(t)

# Solve system with initial conditions
sol = dsolve([eq1, eq2], [x(t), y(t)], ics={x(0):1, y(0):0})

print(sol)
\end{lstlisting}

\textbf{Result:} $x(t) = e^{3t}\cos(4t),\quad y(t) = e^{3t}\sin(4t)$  
\end{frame}

\section{\textbf{Live demo...}}


\begin{frame}{Reading list}
\begin{itemize}
\item C. Thomas Wu, (2010) "An Introduction to Object-Oriented Programming with Java", McGraw-Hill  

\item Priestley, (2003) "Practical Object-Oriented Design with UML", McGraw-Hill 

\item Peter Van Roy and Seif Haridi (2004) "Concepts, Techniques, and Models of Computer Programming", MIT Press. 

\item Haskell, R. \& Hanna, A. (2008) "Introduction to Functional Programming in Python and C++", Cambridge University Press. 

\item Andrew Troelsen and Philip Japikse (2020) "Pro C\# 9 with .NET 5: Foundational Principles and Practices in Programming", Apress. 

\item Axel Rauschmayer (2019) "JavaScript for impatient programmers", Dr. Axel Rauschmayer. 

%\item Boris Cherny (2019) "Programming TypeScript: Making Your JavaScript Applications Scale", O'Reilly Media. 

%\item Brian Goetz (2006) "Java Concurrency in Practice", Addison-Wesley. 
\end{itemize}
\end{frame}

\begin{frame}[fragile]{List of important commands}
\begin{itemize}
\item Create your environment with
\begin{lstlisting}[style=advancedshell]
conda create -f `yml file' -n `env_name' python=3.9 
\end{lstlisting}

\item Activate that env
\begin{lstlisting}[style=advancedshell]
condo activate <env_name>
\end{lstlisting}
\item Install Jupyter
\begin{lstlisting}[style=advancedshell]
conda install -y jupyter  
\end{lstlisting}
\item Add the current env in the notebook kernel
\begin{lstlisting}[style=advancedshell]
python -m ipykernel install --user --name <env_name> --display-name 'kernel_name'
\end{lstlisting}
\item Export your conda in yml file 
\begin{lstlisting}[style=advancedshell]
conda env export | grep -v `^prefix:' > env.yml  
\end{lstlisting}
\end{itemize}
\end{frame}

\section{\textbf{Group Theory with SageMath}}
\begin{frame}[fragile]{Introduction to Group Theory with SageMath}
\textbf{Group Theory} studies algebraic structures called \emph{groups}.

A \textbf{group} is a set $G$ with a binary operation $\cdot$ satisfying:
\begin{itemize}
\item \textbf{Closure:} $a, b \in G \implies a \cdot b \in G$
\item \textbf{Associativity:} $(a \cdot b) \cdot c = a \cdot (b \cdot c)$
\item \textbf{Identity:} $\exists e \in G$ such that $e \cdot a = a \cdot e = a$
\item \textbf{Inverses:} $\forall a \in G, \exists b \in G$ such that $a \cdot b = b \cdot a = e$
\end{itemize}
\textbf{Example in SageMath: Symmetric group $S_3$}
\vspace{1mm}
\begin{lstlisting}[style=mypython]
G = SymmetricGroup(3)
print("Elements of S_3:", list(G))
print("Is S_3 abelian?", G.is_abelian())
\end{lstlisting}

\vspace{-1mm}
\textbf{Output:}
\begin{itemize}
\item Elements of $S_3$: \texttt{[(1,2,3), (1,3,2), (1,2), (2,3), (1,3), ()]}
\item Is $S_3$ abelian? \texttt{False}
\end{itemize}
\end{frame}

\begin{frame}[fragile]{Examples of Groups in SageMath}
\vspace{-1mm}
\textbf{1. Symmetric group $S_3$}
\begin{lstlisting}[style=mypython]
G = SymmetricGroup(3)
print(list(G))              # Permutations of 3 elements
print(G.is_abelian())       # Is S_3 abelian? (False)
\end{lstlisting}

\vspace{-1mm}

\textbf{2. Integers under Addition $(\mathbb{Z},+)$}
\begin{lstlisting}[style=mypython]
Z = IntegerModRing(7)
print(list(Z))              # Elements: 0,1,...,6 (modulo 7)
print(Z.is_commutative())   # True, Z/7Z is abelian under addition
\end{lstlisting}

\vspace{-1mm}

\textbf{3. Cyclic group $C_4$ of order 4}
\begin{lstlisting}[style=mypython]
C4 = CyclicPermutationGroup(4)
print(C4.order())           # 4
print(C4.is_cyclic())       # True
\end{lstlisting}

\vspace{-1mm}

\textbf{4. Quaternion group $Q_8$}
\begin{lstlisting}[style=mypython]
Q8 = QuaternionGroup()
print(Q8.is_nonabelian())   # True
print(Q8.center())          # Center elements of Q_8
\end{lstlisting}
\end{frame}

\begin{frame}[fragile]{Subgroup Operations in SageMath}

\textbf{Let's work with the symmetric group $S_4$:}

\begin{lstlisting}[style=mypython]
G = SymmetricGroup(4)
# 1. List all subgroups (up to isomorphism)
subs = G.subgroups()

# 2. Generate a subgroup from elements
a = G((1,2))        # permutation (1 2)
b = G((3,4))        # permutation (3 4)
H = G.subgroup([a, b])
print(f "Order of H: {H.order()}, Elements:, {list(H)}")

# 3. Normal closure (smallest normal subgroup containing an element)
N = G.normal_closure([a])

# 4. Intersection of two subgroups
H1 = G.subgroup([G((1,2))]); H2 = G.subgroup([G((1,2,3,4))])
intersection = H1.intersection(H2)
\end{lstlisting}

\end{frame}

\begin{frame}[fragile]{Exercise: Subgroup Operations in $S_3$}
Let $G = S_3$ be the symmetric group of degree 3.

\textbf{Tasks:}
\begin{enumerate}
\item List all the subgroups of $G$.
\item Find a subgroup $H$ of order $2$ and list its elements.
\item Is $H$ a normal subgroup of $G$? Justify your answer.
\item Find the intersection of $H$ with a subgroup $K$ of order $3$ in $G$.
\end{enumerate}

\vspace{2mm}
\textbf{Hint:} You can use SageMath commands like\\
\vspace{0.5cm}
\begin{lstlisting}[style=mypython]
G = SymmetricGroup(3)
G.subgroups()
\end{lstlisting}
\end{frame}

\begin{frame}{Usefull tutorials}

\end{frame}

\end{document}
